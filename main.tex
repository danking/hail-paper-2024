
\documentclass[10pt,a4paper%% ,twoside,openright,titlepage,
%%                headinclude,footinclude,numbers=noenddot,
%%                captions=tableheading
]{article}
%% \usepackage[utf8]{inputenc}
\usepackage{url}
\usepackage{parskip}

\usepackage{authblk}
\usepackage{tabularx}
\usepackage{adjustbox}
\usepackage[dvipsnames,table]{xcolor} % needs to come before forest for some reason
\usepackage{forest}

%% \usepackage{amsmath,amssymb}
%% \usepackage{newpxtext,newpxmath}
%% \usepackage{biblatex}
%% \addbibresource{vldb/vldb.bib}
\usepackage{listings}
%\usepackage{inconsolata}
\usepackage[T1]{fontenc}
\usepackage[scaled]{beramono}
%% \usepackage{subfig}
\usepackage{subcaption}

\definecolor{keyword}{rgb}{0.6,0,0}
\definecolor{codegreen}{rgb}{0,0.6,0}
\definecolor{codegray}{rgb}{0.5,0.5,0.5}
\definecolor{codepurple}{rgb}{0.58,0,0.82}
\definecolor{backcolour}{rgb}{0.97,0.97,0.97}

% https://tex.stackexchange.com/questions/297345/why-is-the-start-row-of-rowcolors-ignored-in-tabularx
\newcounter{tblerows}
\expandafter\let\csname c@tblerows\endcsname\rownum


\lstdefinestyle{mystyle}{
    backgroundcolor=\color{backcolour},
    commentstyle=\color{codegreen},
    keywordstyle=\color{keyword},
    numberstyle=\tiny\color{codegray},
    stringstyle=\color{codepurple},
    basicstyle=\ttfamily\footnotesize,
    breakatwhitespace=false,
    breaklines=true,
    captionpos=b,
    keepspaces=true,
    numbers=left,
    numbersep=10pt,
    showspaces=false,
    showstringspaces=false,
    showtabs=false,
    tabsize=2,
    framexleftmargin=5pt,
    framexrightmargin=5pt,
    framextopmargin=3pt,
    framexbottommargin=3pt,
    frame=tb,
    framerule=0pt,
}

\lstset{style=mystyle}

\usepackage[nochapters%% eulermath=false,
%%   beramono=true,
%%   palatino=true,
%%   pdfspacing,
%%   style=classicthesis
]{classicthesis}


%% \newcounter{mparcnt}
%% \renewcommand\themparcnt{\raise0.5ex\hbox{\arabic{mparcnt}}}
%% \newcommand\mpar[1]{\refstepcounter{mparcnt}\themparcnt\marginpar{\footnotesize\themparcnt #1}}


\begin{document}

\pagestyle{plain}

\title{\rmfamily\normalfont\spacedallcaps{The Hail System: Computing for Data-Intensive Science}}

\author[4]{John Compitello}
\author[1,2,3]{Jacqueline I. Goldstein}
\author[1,2,3]{Daniel Goldstein}
\author[1,2,3]{Dan King}
\author[1,2,3]{Timothy Poterba}
\author[1,2,3]{Iris Rademacher}
\author[1,2,3]{Patrick Schultz}
\author[1,2,3]{Christopher Vittal}
\author[4]{Arcturus Wang}
\author[1,2,3]{Edmund Higham}
\author[1,2,3]{Konrad Karczewski}
\author[1,2,3]{Cotton Seed}
\author[1,2,3]{Benjamin M Neale}

\affil[1]{Program in Medical and Population Genetics\\ Broad Institute of MIT and Harvard\\ Cambridge, MA, USA.}
\affil[2]{Analytic and Translational Genetics Unit\\ Massachusetts General Hospital\\ Boston, MA, USA.}
\affil[3]{Stanley Center for Psychiatric Research\\ Broad Institute of MIT and Harvard\\ Cambridge, MA, USA.}
\affil[4]{TBD}

\renewcommand\Authfont{\scshape\small}
\renewcommand\Affilfont{\itshape\scriptsize}
%% \author{
%%   \spacedlowsmallcaps{Daniel King} \and
%%   \spacedlowsmallcaps{John Compitello} \and
%%   \spacedlowsmallcaps{Jacqueline I. Goldstein} \and
%%   \spacedlowsmallcaps{Daniel Goldstein} \and
%%   \spacedlowsmallcaps{Konrad Karczewski} \and
%%   \spacedlowsmallcaps{Timothy Poterba} \and
%%   \spacedlowsmallcaps{Iris Rademacher} \and
%%   \spacedlowsmallcaps{Patrick Schultz} \and
%%   \spacedlowsmallcaps{Christopher Vittal} \and
%%   \spacedlowsmallcaps{Arcturus Wang} \and
%%   \spacedlowsmallcaps{Cotton Seed} \and
%%   \spacedlowsmallcaps{Benjamin M Neale}
%% }
\date{\spacedlowsmallcaps{December 2022}}
\newcommand{\tableheadline}[1]{\spacedlowsmallcaps{#1}}
\maketitle
\begin{abstract}
    Abstract.
\end{abstract}

\tableofcontents

\section{Introduction}

The Hail System is an open-source, multi-tenant, spot-tolerant, elastic, horizontally scalable,
cost-metered workflow engine, relational algebra engine, and linear algebra engine. We describe its
unique composition of capabilities that enable the analysis of very large human genome sequencing
datasets by computational scientists without expertise in distributed systems or data engineering.

\subsection{Genetics and Genomics}

Genetics is the study of inheritance and genomics is the study of the genome. These two fields
endeavor to mechanistically explain the effect on phenotypes of variation in the genome. For
example, in the aughts, researchers discovered that certain mutations in the gene PCSK9 lead to an
increased risk of coronary artery disease and death \cite{Hall2013-us}\cite{Abifadel2003-lb}. Other
researchers discovered healthy adult individuals in which PCSK9 was essentially disabled
\cite{pcsk9-insights}. This evidence lead to the development of new drugs that inhibit PCSK9
providing an alternative to statins and other drugs for lowering LDL levels.

At the time PCSK9 was discovered, the human genome project was just finishing. The assembly and
analysis of just one sequence was a mammoth undertaking in cost and time. In the twenty years since,
we have scaled up from one sequence to one million. This represents a near doubling in the number of
samples every year for twenty consecutive years.

As the rate of sequencing grew, the challenge of deriving insight from a sequencing dataset
transformed from a biological engineering challenge into a software engineering one. The Hail System
represents our approach to tackling these challenges.

\subsection{The Three Phases of Data-Intensive Science}

In 2007, Jim Gray exhorted the computer science community to do a better job of producing tools to
support the whole of scientific research. He identified four paradigms of science: empirical,
theoretical, computational, and ``data exploration'', ``eScience'', or, our preferred term,
``data-intensive science'' \cite{fourthparadigm-chapter-1}. Data-intensive science is characterized
the use of automated machines and computers to capture, process, store, and analyze large
datasets. Experientially, data-intensive science begins where Excel breaks down.

Within genomics, and we hypothesize analogous phases exist in other fields, data-intensive science
has three analysis phases:

\begin{itemize}
\item Primary. The physical-digital interface: sequencers, telescopes.
\item Secondary. Recode technical measurements into model-relevant values: alignment.
\item Tertiary. Interactive \& iterative knowledge making: regression, machine learning.
\end{itemize}

Data flows from the physical world through primary, second, and finally into tertiary analysis. The
product of tertiary analysis is scientific insight, encoded both in research papers and released
datasets. The phases live in conversation with one another. For example, more sophisticated tertiary
analysis may enable primary analysis to use cheaper or higher-throughput methods
\cite{bge}. Moreover, as tertiary analysis advances scientific understanding, secondary analysis
ought to adopt new data representations which reflect this new understanding.

Unfortunately, secondary and tertiary analysis have historically used different computational
primitives. Providing one set of primitives that serves both phases empowers tertiary scientists to
upstream the advances into secondary analysis, thus benefiting all consumers of secondary analysis
products.

\subsection{Genomic Data Types}

A typical human genome has about three billion locations, called loci, across 24 contigs (22 pairs
of autosomes, one pair of sex chromosomes, and the mitochondrial genome). Each locus is identified
by its contig and 1-indexed position on the contig. The human \emph{reference} genome is, in
essence, a three billion character string over the alphabet of \emph{DNA nucleobases} ${A, G, T,
  C}$. An allele is a sequence of nucleobases observed at a locus. An allele is not necessarily
length one; indeed, small and large regions of a particular sequence can be inserted or deleted
relative to the reference. A \emph{variant site} is a locus and the list of alleles observed at that
locus.

A sequenced human genome is represented as a sparse vector indexed by locus. At each locus is an
experimental observation with a point-estimate, the genotype, and quality metadata about that
experiment. A genotype is the tuple of alleles observed at this locus. Sequenced human genomes are
sparse because most genotypes comprise only the reference allele.

Sequenced human genomes are usually stored in the plain-text Variant Call Format (VCF) \cite{vcf4.3},
compressed by Block Gzip, and represented as Genomic VCF (GVCF) \cite{gvcf}, which uses run-length
encoding. The VCF is a tab-separated representation of genome-major, sample-minor matrices of
integers, floats, strings, and arrays thereof. In addition to the elements of the matrix, the VCF
permits an arbitrary number of genomic metadata columns.

Exome sequencing is the process of measuring a subset of the human genome that covers all the
protein-coding genes. An sequenced exome is typically one-hundredth the size of a sequenced
genome. We use the term ``sequence'' to generically refer to a genome or an exome.

\subsection{Genomic Tertiary Analysis}

Tertiary analysis begins by combining one or more sequences into a variant-site by sample
matrix. This operation is effectively transposition. Analysis of this matrix proceeds as follows:

\begin{itemize}
\item Quality control. Identify a set of high quality genotypes, variants, and samples. Quality is
  determined by interrogation of the per-genotype quality metadata; statistical aggregates across
  each variant-site, each sample, variant-regions, and sample groups; and comparison to variant-site
  and sample metadata.
\item Annotation. In addition to their use for quality control, variant-site and sample metadata are
  joined for use in the analysis of phenotypes.
\item Correlation. Many statistical genetics methods assume uncorrelated samples and
  features. Satisfying this assumption requires scientists to first compute variant-site and sample
  correlation matrices and then filter the original dataset to sets deemed sufficiently uncorrelated.
\item Regression analyses. Geneticists are interested in a litany of statistical models including:
  \begin{itemize}
  \item Per-variant-site linear, logistic, \& Poisson regression.
  \item Per-genomic-region linear, logistic, Poisson regression as well ``kernel'' methods.
  \item Mixed models, which extend the above methods with non-diagonal sample covariance.
  \end{itemize}
\item Legacy tools. Genetics and genomics has a large library of specialized and optimized
  tools. They are written in a litany of languages from Java to C++ to Perl to Python to R. These
  tools are almost universally single-machine and require the user to manually partition datasets
  along the genomic axis.
\item Visualization. A regression analysis may yield 30 million to one billion results.
\end{itemize}

Following the example of \cite{evaluating-query-languages-and-systems-for-hep}, we explore a few
simple queries of interest to the genetics and genomics community and their realization in different
query languages. The queries are significantly influenced by the chosen representation. In BigQuery,
we use a ``coordinate'' or ``tall'' representation where each row corresponds to a locus, allele,
sample triplet.

\begin{enumerate}
\item For each variant-site, calculate the allele frequency for each allele.
\item For each variant-site, calculate the allele frequency for each allele for each user-defined
  sample group.
\item For each interval of variant-sites, regress the count of predicted-loss-of-function
  non-reference alleles against the phenotypes.
\end{enumerate}

\begin{figure}[h]
  \begin{subfigure}[h]{\textwidth}
    \lstinputlisting[language=Python]{vldb/query-example-schema-hail.txt}
    \caption{Hail explicit}
    \label{fig:query-example-schema-hail}
  \end{subfigure}

  \begin{subfigure}[h]{\textwidth}
    \lstinputlisting[language=SQL]{vldb/query-example-schema-bigquery.sql}
    \caption{BigQuery}
    \label{fig:query-example-schema-bigquery}
  \end{subfigure}
  \caption{Schema in each system. In Hail, we use a Matrix Table which is a wide
    representation. Each physical record contains an array of entries. In BigQuery, we use a tall
    representation. Each physical record holds one entry.}
  \label{fig:query-example-schema}
\end{figure}

\begin{figure}[h]
  \begin{subfigure}[h]{\textwidth}
    \lstinputlisting[language=Python]{vldb/query-example-one-hail-builtin.py}
    \caption{Hail}
    \label{fig:query-example-one-hail-builtin}
  \end{subfigure}

  \begin{subfigure}[h]{\textwidth}
    \lstinputlisting[language=Python]{vldb/query-example-one-hail.py}
    \caption{Hail explicit}
    \label{fig:query-example-one-hail}
  \end{subfigure}

  \begin{subfigure}[h]{\textwidth}
    \lstinputlisting[language=SQL]{vldb/query-example-one-bigquery.sql}
    \caption{BigQuery}
    \label{fig:query-example-one-bigquery}
  \end{subfigure}
  \caption{Calculating allele-frequencies.}
  \label{fig:query-example-one}
\end{figure}

\begin{figure}[h]
  \begin{subfigure}{\textwidth}
    \lstinputlisting[language=Python]{vldb/query-example-two-hail-builtin.py}
    \caption{Hail}
    \label{fig:query-example-two-hail-builtin}
  \end{subfigure}

  \begin{subfigure}{\textwidth}
    \lstinputlisting[language=Python]{vldb/query-example-two-hail.py}
    \caption{Hail explicit}
    \label{fig:query-example-two-hail}
  \end{subfigure}

  \begin{subfigure}{\textwidth}
    \lstinputlisting[language=SQL]{vldb/query-example-two-bigquery.sql}
    \caption{BigQuery}
    \label{fig:query-example-two-bigquery}
  \end{subfigure}
  \caption{Calculating allele frequencies by sample group.}
  \label{fig:query-example-two}
\end{figure}

\begin{figure}[h]
  \begin{subfigure}{\textwidth}
    \lstinputlisting[language=Python]{vldb/query-example-three-hail.py}
    \caption{Hail}
    \label{fig:query-example-three-hail}
  \end{subfigure}

  \begin{subfigure}{\textwidth}
    \lstinputlisting[language=SQL]{vldb/query-example-three-bigquery.sql}
    \caption{BigQuery}
    \label{fig:query-example-three-bigquery}
  \end{subfigure}
  \caption{Regression of intervals.}
  \label{fig:query-example-three}
\end{figure}

\subsection{The Hail System}

For the past eight years, we have developed the Hail System to enable tertiary analysis of
sequencing datasets. It comprises several components:

\begin{enumerate}
\item Two high-level interfaces.
  \begin{enumerate}
  \item Hail Query. Dataframes supporting relational and linear algebra on tables and matrices.
  \item Hail Batch, a library for defining ``workflows'', directed acyclic graphs of arbitrary
    GNU/Linux binaries.
  \end{enumerate}
\item Hail Scheduler. Elastic, multi-tenant, scalable, container-based cloud cluster scheduler.
\item Cloud-native file API supporting local files and cloud object storage.
\item In-memory analysis-oriented data formats.
\item At-rest data formats.
\item At-rest indexed, partitioned dataset formats.
\item Query planner and optimizer for relational operations.
\item Expression compiler for compiling scalar operations.
\item Spot-tolerant, distributed sort.
\end{enumerate}

Tertiary analysis typically makes use of every component. Specifically:

\begin{enumerate}
\item Relational and linear algebra computes statistics and correlations.
\item Workflows execute critical legacy software massively in parallel.
\item The entire analysis is scheduled across the same multi-tenant cluster. Cost, logs, resource
  usage, and profiling information are all available through a single web user interface.
\item The cloud-native file API is used both in Hail Query data processing code as well as
  client-side to load small results onto the user's machine.
\item Immutable and cheap-to-decode in-memory formats are used.
\item ??? at-rest data
\item Hail's indexed, partitioned formats enable fast subsetting on the genomic axis.
\item Expression compiler compiles Python-like operations into efficient JVM bytecode.
\item Distributed sort powers both joins and grouped aggregations.
\end{enumerate}

\subsection{The Importance of Dataframes}

Dataframe APIs embedded in general purpose programming languages empower users to automate and build
custom abstractions. In particular, the Genome Aggregation Database, the largest public and
non-commercial human sequencing dataset, curates a pair of Python libraries totalling 48,000 lines
of code built on top of the Hail library without the direct involvement of the Hail team.

\subsection{Contributions}

The Hail System is unique primarily for its composition, not its particular components. Few other
systems provide a unified API for workflows of binaries, dataframe-based relational algebra, and
linear algebra on partitioned datasets executed by a serverless backend. Of the major distributed,
serverless data engines, Databricks supports user-provided container images, Snowflake has a closed
beta for executing user-provided container images, and BigQuery has no such support. Databricks has
long supported a dataframe interface. Snowflake introduced dataframes in 2022. In 2023, BigQuery
introduced a pre-GA dataframe API.

As far as we know, the Hail System is the only public and open-source system supporting workflows of
binaries, dataframe-based relational algebra, and linear algebra on partitioned datasets.

\section{Background and Related Work}

\subsection{Workflow Languages, Workflow Engines, and Serverless Engines}

Serverless engines execute tasks at least once. A task is typically defined by an OCI container
image, an execve invocation, and resource requirements.

A workflow engine executes a directed acyclic graph of tasks. Some systems allow tasks to read files
produced by tasks on which they depend. Workflow engines may or may not serverless. Non-serverless
workflow engines either require manual provisioning of VMs or automatically start and stop VMs for
each task.

% Need to assess each workflow and serverless engine to determine for each one if its workflow,
% serverless, or workflow-and-serverless.

There are many workflow languages. A few of interest to the bioinformatics community include: Common
Workflow Language (CWL), Workflow Description Language (WDL), Apache Airflow, Galaxy, and
Nextflow. CWL, WDL, Galaxy, and Nextflow all define custom languages for describing workflows. In
contrast, Apache Airflow simply exposes a Python API.

FIXME: are dags static or dynamic, looping/vector jobs.

There are also many workflow engines. Each of the aforementioned languages has its own workflow
engine. Some of these systems support multiple languages.

Scaling serverless execution systems to thousands of VMs and millions of containers is challenging
enough that Hashicorp and OpenAI have collectively written four blog posts about it. Hashicorp
described using Nomad to schedule one million containers across 5,000 VMs in less than five minutes
\cite{hashicorp1m} and two million containers across 6,100 VMs in 22 minutes and 14 seconds. OpenAI
described scaling Kubernetes-based workloads to 2,500 VMs \cite{openai2500} and then 7,500 VMs
\cite{openai7500}.

\subsection{Analytical Query Languages and Engines}

The bioinformatics community uses many different query languages including Excel, grep, awk, Pandas
DataFrames, R DataFrames, SQL, and Spark DataFrames. Before the last ten years of rapid data size
growth, bioinformaticians largely embraced the Unix philosophy. Use of grep, awk, cut, and paste was
pervasive. More complex analyses were written in C, C++, perl, or R. Since the mid 2010s, trainees
are largely fluent in Python or R, but have rarely written a SQL query.

The Pandas DataFrame interface is the \emph{de facto} standard for dataframes in the Python
ecosystem. Two notable alternatives to Pandas are Ibis, which aims to provide a single interface to
many query engines, and xarray, which generalizes from single-axis, tabular dataframes to
multiple-axis tensorial dataframes.

Commonly used single-machine query engines include R DataFrames, Pandas DataFrames, sqlite, MySQL,
and PostgreSQL. The last ten years have seen an explosion of embedded, single-machine, analytical
query engines (though some optionally integrate with a distribution system). Vaex \cite{vaex},
originally developed for astronomical datasets; Modin \cite{modin}; Velox \cite{velox}, a library
for building query engines; Apache DataFusion \cite{datafusion}; DuckDB \cite{duckdb}, an analytical
analogue of \texttt{sqlite}; and Polars.

The high energy physics community maintains ROOT, a custom, column-oriented query engine suited to
their needs \cite{root}. Physicists prefer to store their data denormalized because the data for a
given ``event'' is always analyzed in the context of that event.

Commonly used distributed analytical query engines include Snowflake, Databricks/Apache Spark,
BigQuery, Azure Synapse Analytics, Amazon Athena, Dask, and Databend.

\subsection{Linear Algebra Engines}

Distributed linear algebra is historically the domain of HPC, in particular OpenMPI-based
solutions. Elemental \cite{elemental} is a relatively new system which revisits the design choices
of ScaLAPACK and PLAPACK in light of hardware developments since the 90s, particularly the rise of
multi-core machines. Development of Elemental has ceased but a fork focused on GPGPUs named Hydrogen
is under active development by Lawrence Livermore National Laboratories \cite{hydrogen}.

Along the lines of rethinking linear algebra as the non-uniform memory hierarchy grows, Smith and
van de Geijn propose a systematic way to derive efficient algorithms for matrix multiplication for
systems with any number of caches \cite{momms}. They note and we agree that: ``Hard drives and other
similarly slow storage devices can be thought of as another layer of the memory hierarchy. Because
of this, we believe the methodology in this paper can be used to instantiate out-of-core algorithms
for MMM.''

Shankar et al. developed numpywren to demonstrate the promise of distributed linear algebra on
serverless engines. A key difference between Numpywren and traditional distributed relational
algebra engines like Apache Spark is the need for directed acyclic graphs of tasks rather than
simple fan-out-fan-in graphs.

\subsection{Relational Algebra on Tensors}

FIXME: GenomicsDB, TileDB, SciDB

\subsection{Partitioned Datasets}

The authors are aware of only one file format (besides Hail's own) with first-class support for
partitioning: Zarr. Zarr stores partitions of the dataset in distinct files, which makes it
appropriate for use with cloud object stores like S3. A Zarr array conceptually is a
multi-dimensional array. The set of possible element types include integers, floats, structures,
strings, and arrays thereof.\footnote{Variable length data types require extra user-specified
configuration.}

Parquet and ORC are efficient column-oriented formats. Tools, such as Apache Spark, support
partitioning a into a collection of Parquet or ORC files. While the Parquet docs note that multiple
Parquet data files could use a single metadata file, in practice, query engines do not create such a
file. As a result, collecting the key bounds or the number of records in each partition requires
$O(N_{partitions})$ object storage requests.

FIXME: TileDB ...

HDF5 is not a partitioned format; however, Highly Scalable Data Service (HSDS) permits the
partitioned storage of HDF5 in cloud object stores. HSDS is usually a separate process which sits
between the object store and the HDF5 client application. There is at least one Python
implementation of HSDS in Python which does not require a separate process.

\subsection{The Essence of the Cloud}

The two essential features of the cloud are spot instances charged with a granularity of seconds and
object storage.

%% various other things I could cite here

%% - File systems / blob storage

%%   - The Google File System (SOSP ‘03)

%%   - Pocket (OSDI 18)

%% - Serverless

%%   - "Occupy the cloud: distributed computing for the 99%." (SoCC 2017)

%% - Query languages

%%   - "Functional Collection Programming with Semi-ring Dictionaries" (OOPSLA 2023)

%% - Vectorized processing versus Compilation

%%   - "Photon: A fast query engine for lakehouse systems." SIGMOD 2022.

%%   - "Everything You Always Wanted to Know About Compiled and Vectorized Queries But Were Afraid to Ask" (VLDB 2018)

%% - Query systems

%%   - "Parallel Database Systems: the future of high performance database systems" DeWitt \& Gray (CACM 92)

%%   - MapReduce 2004

%%   - BigTable 2006

%%   - Dryad (EuroSys ‘07)

%%   - "Interpreting the Data: Parallel Analysis with Sawzall" 2006

%%   - Dremel(https://static.googleusercontent.com/media/research.google.com/en//pubs/archive/36632.pdf) VLDB 2010

%%   - "Resilient Distributed Datasets: A Fault-Tolerant Abstraction for In-Memory Cluster Computing" (NSDI 12)

%%   - "The Datacenter as Computer" 2013

%%   - "Ray: A Distributed Framework for Emerging AI Applications" (OSDI ‘18)

%% - Relational algebra on matrices

%%   - SciDB (SSDBM 2011)

%%   - TileDB (VLDB 2016)

%%   - GenomicsDB

%% - Linear algebra

%%   - Elemental ACM Transactions on Mathematical Software ‘13.

%%   - numpywren (SoCC '20)

%%   - "The MOMMS Family of Matrix Multiplication Algorithms" SC ‘19

%% - Miscellaneous

%%   - "Jim Gray on eScience: A Transformed Scientific Method" 2007(https://languagelog.ldc.upenn.edu/myl/JimGrayOnE-Science.pdf)

%%   - "Science in an exponential world" Szalay \& Gray Nature Commentary 06.

%%   - "Scalability but at what COST?"(https://www.usenix.org/system/files/conference/hotos15/hotos15-paper-mcsherry.pdf) (HotOS 15)

%% - Shuffling

%%   - Hyper Dimension Shuffle: Efficient Data Repartition at Petabyte Scale in SCOPE (VDLB 19)

%% Could I cite Eddie’s Click paper?

%%     - Dryad has this to say:
%%       - Click A similar approach is adopted by the Click modular router 27. The technique used to encapsulate multiple Dryad vertices in a single large vertex, described in section 3.4, is similar to the method used by Click to group the elements (equivalent of Dryad vertices) in a single process. However, Click is always single-threaded, while Dryad encapsulated vertices are designed to take advantage of multiple CPU cores that may be available.

%%     - "Click the Modular Router" Kohler et al. TOCS 2000

%% \section{Terminology}

%% \subsection{Big data type is a partitioned data type. Assumed to be too large to fit in memory on a machine.}

%% \subsection{Small data type is assumed to fit in memory on a machine.}

%% \subsection{A VM is a multi-core computer, with caches, volatile memory, and a network connection.}

%% \subsection{ht is a name bound to a Table}

%% \subsection{mt is a name bound to a Matrix Table}

%% \subsection{bm is a name bound to a Block Matrix}

%% \subsection{DK: typeset version should typographically distinguish between small \& big, python vs IR vs semantics}

\section{Hail Query}

Hail Query has two data types: big data types and small data types. Big data types are too large to
fit in the main memory of a computer. Small data types fit in main memory. There are three big data
types: tables, matrix tables, and block matrices. Big data types have \emph{fields}, identified by
Unicode strings, indexed by zero or more axes, and taking on one value for each present coordinate
of the axes. An axis has zero or more key fields. The values of a key field need not be
unique.

Tables are analogous to SQL tables and dataframes. Tables have two field sets: global fields, which
are indexed by no axes and have exactly one value, and row fields, which are indexed by the row
axis. Matrix tables are analogous to ``wide'' dataframes in which a subset of the columns have the
same data type and logically comprise a two-dimensional matrix. Matrix tables have four field sets:
global fields, row fields, column fields, which are indexed by the column axis, and entry fields,
which are indexed by both the row and column axes. Block matrices are analogous to linear algebraic
matrices. They one unique field, the element field, which must have a numeric data type.

Every small data type includes the unique missing value, has a total ordering, has at least one
in-memory format, and at least one at-rest format. The data types include 32- and 64-bit integers,
32- and 64-bit floats, arrays, sets, dictionaries, structs, tuples, UTF-8 strings, genotype calls,
genomic loci (parameterized by a reference genome), ndarrays, and intervals.

\subsection{At-rest big data type format}

All the big data types use a compressed, indexed, row-key-ordered, row-partitioned, row-oriented
at-rest format. The row-orientation makes decoding fairly expensive, in particular, a
single-threaded Hail decoder does not saturate a 10gbps link.

Tables and matrix tables additionally store two metadata files per field set containing,
per-partition: the minimum and maximum key and the number of records. These metadata files also
include the field set's schema and the at-rest small data type format.

\subsection{At-rest small data type format}

%%     - ORC \& Parquet started ‘13, Arrow ‘16

%%     - Hail relied on Parquet from 2015 to 2017, but migrated off it due to code generation issues. (<https://issues.apache.org/jira/browse/SPARK-18492>, <https://issues.apache.org/jira/browse/SPARK-16845>) When using more than 10,000 columns (it is not uncommon to have a very large number of variant (row) metadata columns), code generation issues arise in Spark’s parquet interface, on which we relied at the time. Due to these issues and a desire to not be at the mercy of third parties, we developed our own format.

The small data type use common at-rest formats such as variable length integers, IEEE 754, UTF-8
strings, and C-like struct layouts. Dictionaries are stored as sorted arrays of key-value
pairs. Sets are stored as sorted arrays. Arrays are sparsely stored with a bitmask indicating which
elements are the missing value.

\subsection{In-memory format}

Big data types, by their nature have no in-memory format. A single partition of a big data type is
represented as a \emph{stream} which is a pull-based, code-generated, lazy sequence of values.

Small data types use common in-memory formats such as two's complement integers, IEEE 754, UTF-8
strings, and C-like struct layouts. Dictionaries are arrays of key-value pairs. Sets are sorted
arrays. Arrays are densely stored with a bitmask indicating which elements are the missing value.

We use region-based\footnote{Also known as arenas or zones.} memory management for every type other
than ndarrays. There are always at least two rows: one per-record row and one per-partition
row. Ndarrays are reference counted because they tend to be large and are used inside loops, whose
body's correspond to a region.

\subsection{Expression language}

The Hail Query expression language is a mutation-free and Turing-complete language supporting all
the aforementioned small data types. Switches, ifs, array maps, array folds, array zips, and loops
are all present. We call the expression language ValueIR and consider it one of the four
sub-languages of the wider IR\footnote{Intermediate representation.} language, described below.

\subsection{Expression language DSL}

The expression language is exposed in Python as an embedded domain specific language which looks and
feels like Python.

\subsection{Randomness}

Hail ensures nearly complete\footnote{Sorting currently involves some randomness. We consider this a
bug.} pipeline execution determinism. Every random sampling function is passed a sequence of
identifiers, starting with a unique static identifier. Samples which appear inside one or more
iteration contexts are passed variable references which are bound by the iteration context to a
dynamic identifier of this particular iteration. These identifiers are passed to counter-based
random number generators, in particular Threefry, to produce a deterministic sequence of
samples. This scheme frees the query planner to change the dataset partitioning and reorder
statements without changing the observed random samples.

\subsection{Relational and linear algebra}

Whereas small data types are manipulated by the ValueIR expression language, large data types are
manipulated by three relational and linear algebraic sub-languages of IR: TableIR, MatrixIR, and
BlockMatrixIR. TableIR roughly matches SQL:

\begin{tabular}{||l l||}
 \hline
 SQL & TableIR \\ [0.5ex]
 \hline\hline
 \texttt{SELECT a, b} & \texttt{t1.select(t1.a, t1.b)} \\
 \hline
 \texttt{WHERE a == b} & \texttt{t1.filter(t1.a == t1.b)} \\
 \hline
 \texttt{X JOIN t2} & \texttt{t1.key\_by(t1.a)} \\
 \texttt{ON t1.a = t2.b} & \texttt{.join(t2.key\_by(t2.b), how='X')} \\
 \hline
 \texttt{SELECT SUM(a)} & \texttt{t1.aggregate(hl.agg.sum(t1.a))} \\
 \hline
 \texttt{SELECT SUM(a) AS a} & \texttt{t1.group\_by(t1.b)} \\
 \texttt{... GROUP BY b} & \texttt{.aggregate(a=hl.agg.sum(t1.a))} \\
 \hline
 \texttt{ORDER BY a} & \texttt{t1.order\_by(t1.a)} \\
 \hline
 \texttt{LIMIT n} & \texttt{t1.head(n)} \\
 \hline
 \texttt{PRIMARY KEY k1, k2} & \texttt{t1.key\_by(k1, k2)} \\
 \hline
 \texttt{HAVING a == b} & \texttt{t1.filter(t1.a == t1.b)} \\
 \hline
\end{tabular}

%%         - Hail uses the term "left join, distinct" to refer to a join where the number of rows in the result equals the number of rows in the left table. This default is useful because scientists rarely want to duplicate rows of genomic data (increased statistical tests!) and frequently assume the right-hand-side of a join has one row per key.\
%%           In SQL this might be written as follows assuming the dialect supports a "FIRST" aggregation which returns the first value in a group\
%%           `select *
%%           from tbl1
%%           left join (
%%               select key1, key2, …, FIRST(col1, …)
%%               from tbl2
%%               group by key1, key2, …
%%           ) as tbl2 on tbl1.key1 = tbl2.key1 AND …
%%           `

%%         - Dictionary index syntax (left join, distinct):\
%%           ht1.select(... ht2 ht1.key …)

%%         - Explicit index syntax (left join):

%%           - Distinct:\
%%             ht1.select(... ht2.index(ht1.key, all\_matches=False))

%%           - Not distinct:\
%%             ht1.select(... ht2.index(ht1.key, all\_matches=True))

The MatrixIR, for manipulating matrix tables, is a two-axis generalization of TableIR. There are two
interesting issues: filtered entries and joins.

\subsubsection{Filtered entries}

Filtering a row (respectively, a column) removes that coordinate of the axis entirely along with all
the corresponding row (respectively, column) fields and entry fields. In contrast, filtering an
entry does not modify the axes at all: it creates a hole in the matrix. Filtered entries are
excluded from aggregations. For example, the 2x2 identity matrix has a mean value of $\frac{1 + 1 +
  0 + 0}{4} = 0.5$; however, when the zero valued entries are filtered out, the mean value is
$\frac{1 + 1}{2} = 1$.

Filtered entries are used by scientists to indicate values that should be treated as "not
observed". Consider the case of genome sequencing. A genotype is derived from measurements called
"reads". Each read is evidence of certain alleles. If we believe that, for a particular genotype,
all the reads are invalid due to some technical error (such as contamination), we would filter that
entry. In contrast, if we have successfully collected reads for a given entry but the collected
reads do not confidently support any particular genotype, we might mark the genotype as missing but
keep the entry in the dataset. In the former case, we assert no observation was made, whereas in the
latter case we keep the observation and choose to model missing data in our downstream statistical
methods.

\subsubsection{Matrix table joins}

There are sixteen possible joins between two matrix tables formed by the product of the four
possible joins on each axis. Hail only provides two joins: left join on rows with an exact match of
columns and outer-join on both axes. In practice, geneticists rarely join two matrix tables
together.

\subsection{Architecture}

Hail Query has two distributed computing backends: Apache Spark and Hail Batch. Hail Query has three
execution components: the \emph{client}, the \emph{driver}, and the \emph{workers}. The client is
often a process on the scientist's laptop. The client to driver communication is backend-dependent,
but usually an HTTP socket. The driver to worker communication is also backend-dependent. In Hail
Batch, the driver writes data into object storage which is read by the workers.

\subsection{Query planner}

Before executing IR, all TableIR, MatrixIR, and BlockMatrixIR nodes eliminated in favor of
ValueIR. Distributed operations are represented in ValueIR as CollectDistributedArray which takes a
list of ``contexts'', a ``global'' value, and a ``body'', which is a function of a context and a
global.

The LowerMatrixIR pass expresses MatrixIR in terms of TableIR. The LowerTableIR pass expresses
TableIR in terms of CollectDistributedArray and other ValueIR. LowerBlockMatrixIR likewise expresses
BlockMatrixIR in terms of CollectDistributedArray and other ValueIR.

BlockMatrixIR cannot be efficiently expressed in terms of TableIR because the former explicitly
refers to blocks by their two dimensional block coordinates. In contrast, TableIR partitioning is
implicitly defined by reads and joins.

%%     - ValueIR contains one distribution mechanism called CollectDistributedArray

%%       - CDA has an array of contexts, a global value, and a function from context and global to some type T.

%%       - Return type is array<T>.

%%       - The body is evaluated at least once on each context value.

%%       - The resulting array contains, in order, the result of evaluating the body on each context.

%%       - Evaluation of a body on a context value (and global value) is called a "job".

CollectDistributedArray requires that body code safely executes concurrently with other executions
of the body code, possibly with the same arguments. This requirement simplifies implementation of
CollectDistributedArray in terms of distributed frameworks which guarantee at-least-once execution
of tasks. For Hail Query, the primary concern is writing to a file name unique to this execution. In
Hail Query, a short random string is included in the filename. Filenames are returned to the driver
which deletes unexpected files produced by duplicate executions.

\subsection{Python Client API is lazy. Operations that force execution are called "actions". Actions include: write, collect, aggregate, and export.}

\subsection{Lifecycle of a query}

    - Client API/DSL generates IR.

    - IR is transmitted from client to driver.

      - In the Spark backend, transmission uses a local HTTP socket between the client and driver processes. We do not use the Py4J socket because it is slow and memory hungry.

      - In the Hail Batch backend, transmission uses cloud object storage as well as a small HTTP request with a cloud object storage URL.

    - Driver iteratively compiles, analyzes, \& optimizes code

      - Compilation steps:

        - Compile MatrixTableIR (now only: BlockMatrixIR, TableIR, ValueIR)

        - Compile BlockMatrixIR (now only: TableIR, ValueIR)

        - Compile TableIR (now only: ValueIR)

      - Analyses and optimizers

        - Type checking

        - Name resolution

        - Constant folding

        - Predicate pushdown

        - Column pruning

        - Inlining

        - Query plan simplification

      - Between each compilation step all analyses \& optimizers are run

    - Driver compiles ValueIR to LIR

      - LIR is a thin layer over JVM Bytecode.

      - LIR has no stack, only registers.

    - Driver splits large LIR methods to avoid JVM method size limitations

    - Driver compiles LIR to JVM Bytecode

    - Driver reflectively loads and executes JVM Bytecode

      - All ValueIR other than CollectDistributedArray are executed on the driver.

      - CollectDistributedArray invokes its body as JVM Bytecode on each context.

        - In the Spark backend, the contexts become RDDPartitions, "globals" are broadcasted values, and the RDD compute method reflectively invokes the body’s JVM Bytecode

        - In the Hail Batch backend, the contexts, globals, and body JVM Bytecode are serialized into cloud storage. One job per context is submitted for execution. When the jobs are complete, the driver concurrently reads each job’s result file.

    - The result is transmitted from the driver to the client.

\subsection{Dataset ordering}

    - Genomics data has a natural row key ordering.

    - Datasets are stored ordered.

    - Joins are always ordered joins.

    - Group by also relies on ordering.

    - Apache Spark’s shuffle operator creates an all-to-all dependency.

      - Every output partition relies on a (possibly empty) set of data from every input partition.

      - To achieve this, every Spark worker sends a message to every other worker.

      - The probability of the entire shuffle failing is equal to the probability of any one worker failing.

      - Suppose the shuffle operation will take ten minutes and the probability of spot instance preemption is one in 1,000. Even a relatively small 100 VM cluster will fail to shuffle one out of ten times.

      - Shuffles mitigation options are limited:

        - Always use non-spot nodes for 5x to 1.1x additional cost.

        - Be aware of every shuffle and change cluster composition for shuffles.

        - Implement a spot-tolerant shuffle.

    - We simply implemented a spot-tolerant algorithm: a distribution sort.

      - The upstream query is executed and written to object storage.

      - Keys are sampled from the written data.

      - Key intervals are chosen to evenly partition the data.

      - Each source partition is split into at most 64 partitions. This inflates the number of partitions by a factor of at most 64. The maximum number of splits is configurable.

      - Partitions are recursively split until all partitions are smaller than 512MB. This number is configurable.

      - Each partition is sorted and written to cloud object storage.

      - Downstream operations read and concatenate the sorted partitions.

\subsection{Hail at-rest format}

Something about the index space being arbitrary types rather than just integers like Zarr.

    - Datasets

      - Row-based

      - MatrixTable and Table

      - Partitioned

      - Always have a row key,  matrix tables have a column key too

      - Table has two components: global (no key) data and partitioned row-keyed data

      - MatrixTable has four components:

        - Global (no key) data (e.g. dataset title, date of collection)

        - Row-keyed data. Partitioned.

        - Column-keyed data. Partitioned.

        - Row-and-column-keyed data. Partitioned in the same way as the row-keyed data.

    - Partitions

      - Sequence of records with "continue" flags

      - Has an Index. A BTree from row key to record.

    - Record

      - ETypes and Buffers define the encoding

        - Etypes describe a format for a given data "type". A function from Type -> bytes. E.g. two’s complement integers, IEEE754 binary representation.

        - Buffers describe a format for bytes. A function from bytes -> bytes. E.g. Zstandard.

\subsection{Hail in-memory format}

    - Fundamentally row-oriented. See future work for thoughts on column-oriented formats.

    - Much like Spark’s Tungsten project, we directly use native memory.

    - Primitives use their usual representations.

    - Structs and fixed-size data types are laid out contiguously. Variable sized data is stored out of line and a pointer is stored inline.

    - We use region-based memory management which is also known as arena or zone based memory management.

      - Both partitions and individual records make for natural regions.

      - A region is a sequence of 64KiB blocks allocated by malloc and a sequence of "large" blocks which are used for allocations larger than 4KiB.

      - Normal and large blocks are not returned to the OS when a region closes. In particular, rows in a partition tend to have similar allocation patterns so we can typically reuse blocks across multiple rows.

      - For certain workloads, we only call malloc on the first row of a partition.

      - Freeing is O(N\_BLOCKS) rather than O(N\_ALLOCATIONS).

    - Instead of region-based memory management, we reference count ndarrays (numerical vectors, matrices, tensors used in linear algebra).

      - Consider gradient descent.

      - Due to Hail’s choice of immutable data structures, a loop implementing gradient descent cannot mutate a vector or matrix.

      - Instead, IR loops have parameters. Arguments from one iteration are passed as the next iteration’s parameters.

      - Each loop iteration is a region, so loop arguments are copied on each iteration.

      - Profiling suggested we were dominated by copying ndarrays for gradient descents.

      - Reference counted ndarray allocations avoid copying the whole ndarray instead only copying a pointer.

\subsection{Shuffles, transposition, columnar storage are all inter-related}

\section{hailtop.batch}

\subsection{Most workflow systems designed custom languages.}

\subsection{In our experience, scientists are familiar with R and Python.}

\subsection{Exposing an API rather than a language empowers users to build the most-useful-to-them abstractions atop our API.}

\subsection{The simplest batch is the empty batch:}

    import hailtop.batch as hb
    b = hb.Batch(backend=hb.ServiceBackend())
    b.run()
    The next simplest batch is a single job printing hello world:
    import hailtop.batch as hb
    b = hb.Batch(backend=hb.ServiceBackend())
    j = b.create\_job()
    j.command("echo hello world")
    b.run()

\subsection{Python is used to automate creation of repetitive jobs.}

\subsection{The lack of higher-level "vector job" abstractions does mean particularly large batches (on the order of 10M jobs) can take tens of minutes to submit.}

\section{Hail Batch}

\subsection{Cost-metered, multi-tenant, spot, elastic, scalable, cloud compute engine.}

\subsection{The smallest unit of the system is a job.}

    - A container image execution. We pervasively assume the existence of bash (or another shell supporting the same semantics).

    - Every job is a member of exactly one batch (described next).

    - Configurable properties:

      - Docker image name

      - Bash script

      - Core-memory ratio (1:7.5, 1:3.75, 1:0.9)

      - Spot-vs-non-spot

      - Disk request

      - "Always Run"

        - Normally a job is canceled if it depends on canceled jobs. "Always Run" jobs when the jobs on which it depends are complete or canceled.

        - These are used to cleanup resources created by previous jobs.

      - Region

        - The cloud-specific region(s) in which a job may execute.

        - Accessing data cross region boundaries is typically non-free.

      - Timeout

        - Maximum runtime of one attempt of the job.

      - Cloudfuse mounts

        - Zero or more buckets to mount using gcsfuse or blobfuse2.

      - Input files

        - A list of cloud storage URLs and local paths to which to download them.

      - Output files

        - A list of local paths and cloud storage URLs to which to upload them.

\subsection{Attempt}

    - Due to network partitions, a job may execute more than once.

    - Each execution has a unique attempt id.

    - Users are charged for every attempt.

\subsection{Batch}

    - A batch contains zero or more jobs.

    - Jobs may depend on zero or more other jobs. There must not be any cycles.

    - A batch is either running, complete, or canceled.

    - A batch starts complete.

    - Jobs only may be added to running or complete batches.

    - A job’s dependencies must be created before it is.

    - Concurrent clients may add jobs to the same batch.

\subsection{Billing project}

    - Every batch is a member of exactly one billing project.

    - Billing projects correspond to a single funded scientific project.

    - Any member of a billing project may view or cancel any batch in that billing project.

\subsection{Database}

    - MySQL database. 4 cores. 16 GiB memory.

    - Three years, >8,000,000 batches, >34,000,000 jobs, \~900 TiB disk.

    - Goals:

      - O(1) API operations: n\_completed jobs in batch, total spend in batch, total spend in billing project, cancel batch.

      - Parallel updates of a job’s information.

      - Consistent view of cost across jobs, batches, and billing projects.

    - What is in it?

      - Batches, VMs, attempts, billing.

      - Job metadata (e.g. dependencies, arbitrary key-value attributes, requested resources) and state (pending, running, canceled).

      - Job commands are stored in cloud storage because the bash script are sometimes large.

    - Rollups

      - The database is a single point of failure and also the main bottleneck.

      - Concurrent inserts or updates to the same row in MySQL either serializes or deadlocks the transactions.

      - In order to allow for parallelism, certain tables have a "token" column which is an integer ranging from 0 to some configurable number, inclusive.

      - Multiple transactions can insert or update rows in parallel.

      - For example, if a user\_job\_states table has an n\_running\_jobs column we would create this table:\
        CREATE TABLE user\_job\_states (\
          user\_id INT NOT NULL,\
          n\_running\_jobs INT NOT NULL DEFAULT 0,\
          token TINYINT\
        )\
        When a job starts running we might execute:\
        INSERT INTO user\_job\_states (user\_id, n\_running\_jobs, token)\
        VALUES(1, 1, FLOOR(RAND() \* 200))\
        ON DUPLICATE KEY UPDATE n\_running\_jobs = n\_running\_jobs + 1\
        When a job completes we might execute the same query with -1 in place of 1.

      - The probability of collision is given by the birthday problem.

      - We currently use 200 but this frequently leads to (retryable) deadlocks.

      - Need to balance the cost of aggregation for read against write parallelism.

    - Billing

      - Provides user feedback and informs the cost limiter when a billing project is overspent.

      - Resource use "rolls up" from attempt to job to batch to billing-project-\&-user-\&-date to billing-project-\&-user.

      - MySQL triggers ensure the rollup is recalculated after every database change.

      - All tables use tokens except the attempt and job tables.

    - Active resources

      - Very large batches (e.g. a 10M job batch) cause the auto-scaler to aggressively add machines. We want a canceled batch to immediately stop triggers scale up, so we need O(1) update to the number of pending jobs.

      - The number of pending and running jobs per pool and per-batch-per-pool are stored in tables using tokens.

      - The autoscaler and scheduler use the pending and running jobs per pool.

      - Cancellation updates the per pool table based on the per-batch-per-pool table in O(N\_POOLS) time.

\subsection{Available resources:}

    - Pools

      - Three core to memory (in GiB) ratios: 1:7.5, 1:3.75, 1:0.9.

      - Spot and non-spot.

      - Disk space.

        - In GCP, disks are dynamically attached when more disk is needed.

        - In Azure, large disk requests fail.

    - Private instance

      - Effectively a shim over the underlying VM API.

      - Use-cases

        - A set of jobs otherwise amenable to standard core ratios but needing one initial or final job with a very large RAM ratio.

        - Users wanting GPUs (which are not currently supported in any pool).

\subsection{Architecture}

    - MySQL database is a centralized state store.

    - Front-end is a horizontally scalable HTTP API for submission and monitoring.

    - Driver is a centralized scheduler and autoscaler.

    - Workers are VMs.

    - A separate auth service is a thin layer over cloud IAM.

    - Only relies on three simple, robust, and pervasive cloud APIs: VM, Object Storage, and Container Image Repository. (VERTICAL INTEGRATION!)

\subsection{Single threaded Python driver can manage about 90 completing jobs per second.}

    - When a job completes, a worker sends an HTTP request to the driver.

    - The driver marks the job as complete in the database.

    - To schedule another job, the driver makes an HTTP request to a worker.

    - Scheduler is triggered either once a minute or when a job completes

      - Scheduler fetches a subset of ready jobs from database for each users

      - Cores are shared fairly among all users with ready jobs.

\subsection{90 completing jobs per second}

    - If job latency is normally distributed with mean 5 minutes, then the driver can handle a steady state cluster of 27,000 jobs. If each job uses one core, that’s a \~1,700 VM cluster.

    - Longer mean job latency permits larger clusters.

    - We’ve operated clusters as large as 5,000 16 core VMs.

\subsection{Cost-metering}

    - Due to the cost constraints on science, enforcing cost limits is important. We also do not want to kill a running job before the user exceeds a cost limit.

    - Our approach leaves a lot to be desired but works well enough at our scale.

    - Once a minute, a worker sends an HTTP request informing the driver of the list of still running jobs. In response, the driver updates the total cost for this job, the batch containing the job, and the billing project containing the batch.

    - Every ten seconds the driver looks for billing projects that are overspent. All running batches in an overspent billing project are canceled which kills the running jobs.

    - Assessment:

      - A network partition in principle, allows arbitrary overspend.

      - Under normal conditions, overspend could be as high as:\
        16 cores \* 70 seconds \* N\_VMs \* usd\_per\_core.

      - Total cost of jobs, batches, and billing projects monotonically increases. The system (and by extension the user) may underestimate cost but never overestimates it.

      - In practice, overspend is usually less than a few dollars and limits are thousands of dollars.

\subsection{Multi-tenant}

Each user job runs in an OCI container started by crun using cgroupsv2.

Each job is in a private network. CPU is controlled by cpu.weight. Memory is controlled by memory.max and memory.high. Swap is prohibited.

Multi-tenancy both mitigates fragmentation (instead of two isolated users fragmenting their own clusters, the users share a single less fragmented cluster) and reduces latency for small \& fast jobs (because, in practice, clusters are often fragmented).

\subsection{Spot tolerant}
A user’s job runs to completion at least once.

Users must write jobs which are idempotent. The Hail Python API facilitates downloading inputs, computing on local files, and then uploading results, which ensures that jobs that do not mutate external resources are idempotent.

\subsection{Elasticity}

Every fifteen seconds, the driver calculates the demanded cores by region by pool by user and starts new instances to meet demand.

There are two scarce resources: cores already leased from the cloud and VM creation operations. The latter is limited by cloud-imposed limits on VM creation operations per minute.

The driver fairly shares these scarce resources across all users accounting for both jobs ready to run and jobs already running.

\section{Hail’s tenets}

\subsection{Pervasive preference for spot instances}

\subsection{Vertical integration: Only rely on slow moving extremely robust third parties (e.g. the Linux kernel)}

\subsection{Scientists must not be subjected to operational work (cite Occupy the Cloud)}

\subsection{Client-Driver-Worker, adopted from PySpark, leverages the cloud but preserves the highly customizable local environment when data is small (i.e. transferred to the client).}

\subsection{Dataframes uber alles, SQL is not well suited to the needs of scientists (cite query languages for hep paper)}

\subsection{The only durable storage is cloud object storage.}

\section{Funding \& Organizational Challenges}

\subsection{Contributions require caching in immense knowledge of the system, in practice we see few open source contributions.}

\subsection{Pharma and BioTech companies use and benefit from the system, but, as of publishing, none have contributed engineering resources or funding.}

\subsection{Elemental was shutdown due to lack of support for the developers}

\section{Qualitative impact on science}

\subsection{Hail Query and the Hail VDS (cite preprint) have been used to combine as many as 955,000 exomes (cite gnomAD v4) and 350,000 genomes (cite All of Us Echo release) into analysis-ready VDSes. Exomes or genomes are sequenced and aligned per-sample. Tertiary analysis prefers a variant-major (aka feature-major) dataset; therefore, producing a VDS from sequences is essentially a very large matrix transpose. See (cite preprint) for details on this process and the VDS format.}

\subsection{Hail Query (historically atop the Apache Spark backend, but work is shifting to the Hail Batch backend) has powered the analysis of several large sequencing studies including Centers for Common Disease Genomics, gnomAD v2, v3, and v4, All of Us 2023 Data Freeze, Schizophrenia Exome Sequencing Meta-analysis. In all, the repository, github.com/hail-is/hail has been cited 180 times.}

\subsection{Hail Batch enabled a multi-ancestry analysis of 7,221 phenotypes, across six continental ancestry groups, for a total 16,119 genome-wide association studies, each study regressing the phenotypes against 30 million genomic loci. This required executing twelve million jobs across a cluster of 80,000 cores totalling over four million core-hours.}

\subsection{Hail Batch provides a qualitatively different experience to users than competing workflow systems. One user had this to say:}

    I had to present today and wouldn't have been able to if Hail Batch hadn't processed all 3,100 genomes in <1hr total which was more than 10x faster than my previous setup. It literally took me less time to learn Hail Batch, port the workflow to it, and have it run on all data than it took to wait for it to finish elsewhere. I was calling tandem repeat expansions, so we now have some candidate diagnoses to pursue.
    The development experience is also incomparably better with local mode and Python tooling.
    All around, this feels like a game changer.

\section{Quantitative evaluation}

\subsection{Single threaded}

    - Table decoding

      - I expect at least 50MiB/s.

      - Not great but a 16-core VM can manage 6.4 gigabits, \~60\% of 10gb

    - Matrix table decoding

    - VDS decoding

    - Table \& matrix table sorting

    - Linear regression rows

    - Table group-by-aggregate

    - gnomAD frequency calculations

\subsection{Multi-threaded}

    - As above.

\section{Quantitative and Qualitative Comparisons}

\subsection{MySQL (as a representative of the traditional RDBMS)}

\subsection{sqllite}

\subsection{BigQuery}

\subsection{Apache Spark}

    - Designed for non-spot instances; ergo, fundamentally not cloud native.

    - Users must either operate their own cluster or use a proprietary serverless Spark platform (e.g. Dataproc Serverless, EMR Serverless).

\subsection{DuckDB}

\subsection{AWS Fargate}

\subsection{Cloud Run}

\subsection{Azure Functions}

\subsection{Numpywren}

\subsection{Ibis}

    - Their Python API/DSL looks more like ours

\section{Future Work}

\subsection{differentiable programming}

\subsection{import/export Arrow for zero-transcoding data transfer from one process to another.}

\subsection{Columnar storage}

    - 50 MiB/s is peak single-threaded decode speed, largely due to the amount of instructions necessary to decode, for example, variable length integers or homogeneous arrays of structures.

    - Columnar storage should enable much higher bandwidth decoding and encoding.

    - Pipelines only reading the genotype field (typically 4 bytes out of 10s) tend to be bottlenecked by the network due to the vast amount of unnecessary data.

    - Repetitive fields should benefit from run-length compression.

\subsection{Vectorized processing}

\subsection{Replace SQL database}

    - Bad things:

      - Token scheme required to allow parallel updates to hot rows

      - Must write in MySQL SQL rather than Python, C++, or Java.

      - No event queue or other way to notify another process of changes.

    - Not sure what’s better though.

\subsection{Tighter cost controls}

    - A better solution would have each worker lease, say, five minutes worth of the cost of a job.

    - Ensures no overspend but would sometimes cancel or prevent from running an otherwise acceptably costing job.

CollectDistributedArray should support dependencies between contexts. This allows for DAGs necessary to express distributed linear algebra (cite numpywren).

A "distinctly keyed" dataset is one where there is at most one row per key value. Hail often works with foreign files and thus has no statistics about the keys. Moreover, Hail does a poor job of opportunistically tracking this information. In theory this information admits faster joins (once a key is processed it is done) and permits warning a user that Hail’s default "left join, distinct" join will elide information in the right-hand-side table.

Add TableMaterialize, MatrixTableMaterialize, BlockMatrixMaterialize nodes. These nodes are explicit statements that computing the dataset should be "cheap". The query planner is free to either implement these as a write-read to temporary storage or to eliminate them entirely. Moreover, the query plan can choose encodings best suited for a write-once-read-once.

    - For example, sorting a partitioned dataset requires sampling the new keys and then distributing the data into new partitions. In order to avoid computing the upstream query twice, sort explicitly writes the dataset. For computationally cheap queries, the cost of I/O dominates. In this regime, reading, writing, and finally reading twice is twice as slow as just reading twice.

\subsection{Faster storage.}

    - Cloud storage is impressively scalable but cannot rapidly serve a single large object to 5,000 VMs.

    - Container image pulling suffers similar problems.

    - Cloud storage operations are not free.

    - We want to use a fast read-through cache for objects read by large numbers of Hail Query partitions.

\subsection{Container image extraction is also shockingly slow.}

    - We would like to use a pre-extracted format amenable to lazy loading.

    - Perhaps a torrent style distribution system.

\section{Acknowledgments}

NEU: DVH, Olin, Matthias, Aaron Turon, Dimitrios Vardoulakis, Dee, others? Harvard: Steve, Greg, Eddie, Scott, Dan, Andrew, others? Broad: Ben, Cotton, Mark, DMac, Hail team past and present, Konrad, KC, gnomAD, Andrea G, AUS, Michael F, Leo G, others? Funders: MSFT, Wertheimer, Stanley, NIH(?), NHGRI(?), others?

\section{Figures}

\subsection{Job <- Batch <- billing project}

\subsection{VDS}

Two MTs, one with column-sparse reference block runs and one with row-sparse variants

\subsection{MT}

The four pieces

\subsection{hailtop.batch example}


\bibliographystyle{ACM-Reference-Format}
\bibliography{vldb/vldb}


\end{document}
